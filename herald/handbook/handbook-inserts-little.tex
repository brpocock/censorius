\batchmode
\documentclass[9pt,twoside,openright,final,article]{memoir}
%% ,letterpaper
\setstocksize{8.5in}{7in}
\settrimmedsize{\stockheight}{\dimexpr 7in-15mm}{*}
% \settypeblocksize{542.024925pt}{415.5524425pt}{*}
\usepackage[top=.5in,left=.25in,right=.25in,bottom=.5in]{geometry}

\usepackage[T1]{fontenc}
\usepackage{fixltx2e}
\usepackage{graphicx}
\usepackage{longtable}
\usepackage{float}
\usepackage{wrapfig}
\usepackage{rotating}
\usepackage[normalem]{ulem}
\usepackage{amsmath}
\usepackage{textcomp}
\usepackage{marvosym}
\usepackage{wasysym}
\usepackage{amssymb}
\usepackage{multicol}
\usepackage{pdfpages}
\usepackage{pifont}
\usepackage{newcent}
\usepackage{rotating}
\usepackage{needspace}
\usepackage{xltxtra}
\usepackage{fontspec}
\usepackage{xunicode}
\defaultfontfeatures{Scale=MatchLowercase}
%% \setromanfont[Numbers=Uppercase]{Hoefler Text}
\newfontfamily\beltanefamily{Beltane}
\newfontfamily\hminfamily{TEGScript}

% \def\hminfamily{}

\def\fpg{{\beltanefamily FPG\ }}
\def\teg{{\hminfamily TEG\ }}
\def\FPG{{\beltanefamily Florida Pagan Gathering\ }}
\def\TEG{{\hminfamily Temple of Earth Gathering\/\ }}
#include "festinfo.tex"

\tolerance=1000
\clubpenalty=70000
\widowpenalty=30000

\hyphenpenalty=7000 \exhyphenpenalty=1000 \pretolerance=1000

\makeheadrule{headings}{\textwidth}{1pt}
\makefootrule{headings}{\textwidth}{1pt}{4pt}

\date{\today}
\title{\festseason{} \festyear{}}
\author{Temple of Earth Gathering \\ Florida Pagan Gathering}
\renewcommand{\pfbreakdisplay}{%
  \needspace{24pt}%
  \vspace{8pt}\\\ding{76}\quad\ding{77}\quad\ding{78}\\%
  \vspace{11pt}}
\setsecnumdepth{none}
\makeevenhead{headings}{\rightmark}{}{\FPG}
\makeoddhead{headings}{\TEG}{}{\rightmark}
\makeevenfoot{headings}{\thepage}{}{\festseason{} \festyear{}}
\makeoddfoot{headings}{\festseason{} \festyear{}}{}{\thepage}

\setaftersubsubsecskip{-1em}

\setsecheadstyle{\beltanefamily\Huge\raggedright}
\setsubsecheadstyle{\large\raggedright}

\let\oldsection=\section
\renewcommand{\section}[1]{%
  % \filbreak
  \vspace{3pt}%
  \needspace{1in}%
  { \hrule } \nopagebreak %
  \begin{center}\oldsection{#1}\end{center}\nopagebreak{}}
\let\oldsubsection=\subsection
\renewcommand{\subsection}[1]{%
  \vspace{6pt}%
  \needspace{1.25in}%
  \begin{center}\textbf{\Large \beltanefamily #1}\end{center}

  \nopagebreak}
\let\oldsubsubsection=\subsubsection
\renewcommand{\subsubsection}[1]{%
  \vspace{1pt}\needspace{1.5in}
  {\large ~~~\beltanefamily #1~~~\ }
  \nopagebreak}


\setlength{\droptitle}{2in}
\setlength{\bottomsectionskip}{0pt}
\setlength{\columnsep}{16pt}
\normalbottomsection{}

\begin{document}

\includegraphics[height=9.75in]{cover.png}

\begin{center} {\beltanefamily \HUGE Florida Pagan Gathering }\texttrademark{} \end{center}

\begin{center} {\beltanefamily \HUGE \festseason{} \festyear{} } \end{center}

\vfill

\vfill

\begin{center} {\footnotesize
    { \hminfamily presented by \TEG Inc}\texttrademark{} } \end{center}

\begin{multicols}{3}

  {\footnotesize \tableofcontents* }

  \columnbreak

  ~ \\
  \vfill

  \begin{center} {\includegraphics[width=.5\linewidth]{teg-fpg-logo.png}} \\
    {\texttt{http://FlaPagan.org/} } \end{center}


\end{multicols}

\pagestyle{empty}
\newpage
\thispagestyle{headings}

\begin{center} {\beltanefamily \HUGE Welcome to FPG \festseason{} \festyear{}} \end{center}
\addcontentsline{toc}{section}{Welcome}


\begin{multicols}{2}


  Welcome to the \FPG\texttrademark{} for \festseason{}, \festyear{}.

  Twenty years  ago, in 1995, the  Church of Iron Oak  began the Freedom
  Fest; initially, to help defray  legal expenses from fighting the City
  of Palm Bay for religious freedom.

  That   Festival    continued   to   grow,   and    under   the   \TEG,
  Inc\texttrademark{}\footnote{The \FPG,  \fpg, \teg, the  Blowing Palms
    and  Temple  logos  are  all registered  trademarks  of  \TEG,  Inc.
    All rights are  reserved and our trademarks may not  be used without
    permission.}, became today's \FPG.


  \section{Safety}


  \begin{center}
    {\large \textsc{In case of emergency or injury} \\
      \textbf{contact the nearest Guardian \\
        or Staff Member}}
  \end{center}


  The well-trained team of volunteer Guardians of the Community, led
  by the fearless Sqrl, patrol the Festival and help ensure our
  safety.  They are able to administer basic first aid, and do have
  some basic medical supplies.

  \textsc{Pay attention} to any instructions (or sternly-worded advice)
  from the Guardians!

  Your medical health and safety are your responsibility.  Bring
  any medications or supplies you may need. Wear sunblock, and drink
  plenty of water.  Sunburn and dehydration happen even in cool
  weather.

  Check for ticks daily. Ticks like warm places, such as behind knees,
  under arms, and in elbows, but can be found anywhere on the body.

  Keep an eye out for the Guardians' fund-raising booth as
  well. Remember, \emph{Guardians are sexy}!

  \subsubsection{``File of Life''} If you have any medical condi-
  tions that the Guardians or other emergency personnel might need to
  know about in case of an emergency, please pick up a ``File of
  Life'' form from Registration. This is a great tool for you to keep
  pertinent medical history on hand in case you become unconscious or
  unable to speak. Keep this form on you in an easy-to-find
  location. It is also be a good idea to let someone close to you know
  where it is, so they can help.

  \subsubsection{In case of Hazardous Weather} such as strong winds or
  a tornado: If you are camping, in an RV, or staying in Cabins A-D,
  come to the Blue Room with a snack and drink, and a book or
  something to do (you may be there for several hours). If you are
  staying in Cabins E-J or a Lodge, stay there until you hear an
  announcement.


  \subsection{Pool Rules}

  \begin{itemize}
  \item \textsc{No Swimming} without a Camp Lifeguard present.
    Treat the Lifeguard
    with respect at all times. Swim at your own risk.
  \item Follow all \fpg and campground rules. See the signs posted
    near the pool.
  \item No one is allowed on the pool deck after dark.
  \item Wear a bathing suit --- no cut-offs or regular clothing.
  \item Adults must accompany children. All non-potty-trained children
    must wear a swim diaper, or cloth diaper with tight-fitting rubber
    pants.
  \end{itemize}

  \subsubsection{Pool Hours}

  #include "insert-pool-hours.tex"

  % \end{multicols}

  \pagestyle{headings}
  \thispagestyle{headings}
  % \newpage

  % \hline

  \section{Getting Around}

  % \fancybreak{\pfbreakdisplay}

  \subsubsection{Taxi/Trolley Hours}

  #include "insert-trolley-hours.tex"

  % \begin{multicols}{2}

  \subsection{Taxi Service}

  If you need personal door-to-door service, then call for a
  Taxi. Just ask any Staff Member to radio a Taxi to pick you up. You
  can also schedule a Taxi pick-up ahead of time.

  \subsection{Trolley}

  The \fpg Trolley runs along a continuous circuit. Wait at any
  Trolley Stop and the Trolley will pick you up and drop you off at
  any other Trolley Stop. Trolley Stop signs are distinctive vertical
  signs that are bright green on the side facing the approaching
  trolley and bright pink on the back. (Check the map for the route.)

  Minors may \emph{not} ride the Trolley without a parent present.

  No food, beverages ({\small unless in a closed container}), nor
  smoking are permitted on the Taxi or Trolley (or any other \teg
  carts).

  % \fancybreak{\pfbreakdisplay}


  % \columnbreak

  \subsection{Personal Vehicles}


  Driving personal vehicles through camp is not permitted after
  arrival.  Electric personal mobility devices are allowed.

  Watch for Campground and \teg service vehicles and golf carts, or
  emergency vehicles.

  The road is one-way, clockwise, through the camp. From the Main
  Gate, turn left.

  \subsubsection{Parking} After check-in, all guests will have 45
  minutes to locate their cabin or campsite and unload. After this,
  all vehicles must go to the parking lot. Park in neat rows, and do
  not block anyone else's vehicle.  Leave your \fpg vehicle card in
  your windshield, in case wo need to

  If you leave during Festival, stop at the front gate on your way
  back in, and ask a staff member to call a Taxi to meet you at the
  parking lot; or, catch the Trolley at the Trolley Stop near the Main
  Gate to convey you back to your campsite or cabin.

  Avoid the parking lot after dark. If you must go, have someone join
  you. Bring a flashlight or lantern, and be aware of wildlife
  hazards.

  \subsubsection{Forest} Our new site has an amazing wooded area for
  camping. Campers are not allowed to bring their vehicles into the
  camping area after dusk. Campers may be asked to unload their
  vehicles at the edge of the Forest and carry items to their sites.

  \subsubsection{Handicap Parking} Each vehicle must have a
  State-issued Handicap Placard or License Plate or a \fpg issued
  Parking Tag\footnote{If you require handicap parking, but do not
    have a state-issued plate, see Registration at the Main Gate.} to
  park in the Handicapped Parking area.


  \subsubsection{RV Camping Area} No tents are permitted in the RV
  Camping area. Once your RV, pop-up camper, van, or travel trailer is
  parked in the RV camping area it may not be moved until end of
  Festival.

  % \end{multicols}


  \section{Communications \& Staff}

  % \begin{multicols}{2}


  %   \fancybreak{\pfbreakdisplay}

  %   \fancybreak{\pfbreakdisplay}

  \subsection{Ministry of Magic {\small $($Staff Office$)$}}

  Our office for the Festival is located between the Blue Room and
  Dining Hall.

  Our Office staff are happy to assist you with finding information,
  arranging Taxi service, or locating a staff member to assist you.

  \subsubsection{Shirts, Totes, Mugs, Bottles} You may pick up
  pre-ordered \fpg merchandise, or make a purchase, starting Thursday
  morning. Unclaimed merchandise is placed in storage until the next
  Festival.


  \subsubsection{Community Service} For each volunteer shift that you
  perform, you'll earn another entry into a raffle for prizes that
  include half-off or a free admission to your next \fpg!

  Sign up with Registration at the Main Gate.


  \subsection{Emerald City  {\small $($Board of Directors$)$}}

  Got a burning question, comment or suggestion?  Do you just want to
  sit and have a good conversation about \fpg?  Then head on over to
  Emerald City and walk right in.  Grab a chair and sit down in the
  kitchen for a nice chat with any of our friendly Board Members.

  Is there a problem?  Did something happen?  Do you need some advice?
  Emerald City is the place to go for these issues as well...even at
  4am!  Just come on in and knock on the inner door --- it's never too
  late or too early to get some help!

  {\footnotesize We ask that the hour before meal times and during meal times
    are for emergencies only.  We need the time to set up for meals,
    feed our guests and take a moment to eat as well.}

  \subsection{Radios \& Walkie-Talkies}

  Having personal radios on site to help keep track of your family,
  especially your children, is an excellent idea! Set them to
  \textbf{channels 11 and above}; channels 10 and below are in use by
  the \fpg Staff.

  % \fancybreak{\pfbreakdisplay}

  \subsection{\fpg Community Center}

  The Community Center is located on the lagoon side of the Blue Room
  in a screened porch.  There are tables and chairs, where you can
  gather, meet others, or just hang out. Some games will be set out,
  and a Community Bulletin Board where you can post announcements
  (including business and service advertisements).

  Be respectful of others using the space, and clean up after
  yourself.

  A Charging Station for your electronic devices is in the Community
  Center. {\small You're responsible for keeping an eye on your
    devices while they charge.}

  There is a self-serve coffee and tea station with a Karma Jar to
  contribute funds for the purchase of supplies. (In addition to
  coffee, there are teas, sugar and sweetener, and creamer.)  We
  encourage anyone to contribute teas, cocoa, sweeteners, \&c. You
  must bring your own cup!

  % wasting space and looking pretty
  %% {\centering  \includegraphics[width=.75\columnwidth]{fairy.jpeg}}

  \section{Community Organizations}

  % \begin{multicols}{2}

  %   \subsection{Red Tent}

  %   \begingroup
  %   \setlength{\columnsep}{6pt}%
  %   \begin{wrapfigure}{l}{.5\columnwidth}
  %     \includegraphics[width=.5\columnwidth]{red-tent.png}
  %   \end{wrapfigure}

  %   Open all day for women and girls with their mothers; A great place
  %   to relax, meet newcomers and renew friendships.

  %   All who identify as girls or women are welcome to come share coffee,
  %   cookies and conversations in a beautiful and comfortable space
  %   filled with solidarity and great donated stuff for the benefit of
  %   the RT. No need to buy anything- treats and workshop materials are
  %   free. Just hang out, share advice, browse our library, read cards,
  %   eat chocolate and laugh a lot. Bring a chair if you can.

  %   \endgroup

  %   {\small
  %   #include "handbook-red-tent.tex"
  % }

  \subsection{Art Abandonment Project}

  The faeries are at it again! Keep your eyes open for random gifts of
  love just for us at \fpg!

  If you find these little packages of fun marked ``\textsc{fae gift}
  thanks to The Art Abandonment Project,'' they are yours to keep,
  trade or give to a loved one or friend!  All gifts will be randomly
  found around the campsite over the week --- except Vendors' Row
  (which is off-limits) \ldots Happy \festseason{} from the Fae!


  \subsection{Father Sky Lodge}

  \begingroup
  \setlength{\columnsep}{6pt}%
  \begin{wrapfigure}{r}{.4\columnwidth}
    \includegraphics[width=.4\columnwidth]{FSL.png}
  \end{wrapfigure}

  Father Sky Lodge (FSL) is a group effort of Pagan men to provide
  support and resources to their community in order to encourage the
  exploration of Male Spirituality, the Male Mysteries and to return The
  Masculine Divine/God to a place of balance within Neopagan
  spirituality. It allows Pagan men to come together in fellowship and
  brotherhood.

  Our goal is to create a place where anyone, man or woman, can come
  with questions about masculine Pagan Spirituality, and know that
  there will be answers; a place where any member of our Pagan
  community can freely ask ``Can you help me?'' on matters related to
  Pagan Men, and know that the answer will always be a resounding
  ``\textsc{Yes}!''

  We are a brother group with Sisterhood of the Shield and work with
  them to support each other in our Spiritual journeys.

  \endgroup


  \subsection{Sisterhood of the Shield}

  \begingroup
  \setlength{\columnsep}{6pt}%
  \begin{wrapfigure}{l}{.4\columnwidth}
    \includegraphics[width=.4\columnwidth]{sts.png}
  \end{wrapfigure}

  Sisterhood of the Shield (StS) is a women's-only group. Though we
  embrace the traditional aspects of womanhood, we also seek the more
  primal and earthy energy; a place where the warrior women and other
  such traditions are as welcome as the maiden, mother and crone.

  We have an Equalist mentality: Men and Women are equal but there are
  things that each experience that only someone of their own gender can
  understand.

  We are a sister group with Father Sky Lodge and work with them to
  support each other in our Spiritual journeys.

  \endgroup

  \section{Children \& Minors}

  \fpg welcomes families of all  kinds. Children and minors are welcome.
  Adults who wish to bring children of friends and family may do so, but
  must provide a  copy of a notarized  Letter of Guardianship\footnote{A
    blank  form is  available  from \texttt{register@flapagan.org}.}  at
  check-in.

  Parents (or  guardians) are fully  responsible for the  supervision of
  the  children and  minors  in  their charge,  and  their behavior  and
  conduct  at all  times. Parents  must always  know the  whereabouts of
  their minors. Do not leave Children under 12 unattended.

  Children's and youth activities have a minimum of two adults.  Teen
  Forge is for ages 13-17. It is not staffed by adults, but there is
  an adult adviser.


  \subsection{Kids' Realm}

  Drop off your kids at the  Arts and Craft Room for supervised playtime
  and workshops while  you set-up your campsite, go  shopping on Vendors
  Row, attend your own workshops, or just relax with friends! Be assured
  in the  knowledge that  your precious little  witches and  wizards are
  safe, happily playing together, and  learning from some of our awesome
  presenters. Pick up your children \emph{at 5pm promptly}.


  #include "insert-kids-realm-hours.tex"

  Kids'  Realm is  for  children  who are  potty-trained,  up  to age  12.
  Advise Kids'  Realm staff  of any  allergies, etc.  that your  child may
  have. Also  make sure to apply  sunscreen or bug spray  to your children
  before dropping them off. \fpg does not supply them.

  In case  of inclement  weather, Kids'  Realm will  move into  the Dining
  Hall; pick them  up there. When dropping off your  child, tell the staff
  where to reach you if you're needed.

  If your  child attends Kids'  Realm, \emph{at least  one parent/guardian
    must volunteer for 2 hours at  Kids' Realm.} We cannot do this without
  you! Sign up for your shift the first time you drop off your child(ren).

  \subsection{Tween Time}

  'Tween-agers have their special place  to hang out and express themseves
  at Festival, too.


  \subsection{Teen Forge}

  The Teen Forge is a place set aside for teens, 13-17. The tent is
  yours to use. Come and go as you wish. Bring snacks and soft drinks
  to share. Bring your ideas, your complaints, your sense of humor,
  and your friendship. There are also workshops and activities for you
  throughout the weekend. Are there activities you wish to have next
  festival? See the Teen Leader or Sage to make suggestions.

  \subsection{Rules of The Forge}

  {\small
    \begin{enumerate}
    \item Do not disrespect each other. {\tiny (Joking is ok.)}
    \item If you break it, fix it; if you can't fix it, pay for it.
    \item If you borrow anything, tell the Teen Leader and give it
      back the next day
    \item If this becomes too much like work, it's time to do
      something different.
    \item Be excellent to each other.
    \item  Don't  do  anything   Dumb\footnote{``Dumb''  is  defined  as
        anything that requires more than \$20 or 20 minutes to fix.}.
    \item All \fpg festival rules apply.
    \end{enumerate}
  }


  \section{\fpg Traditions}

  \subsection{Celtic Games} \label{celticgames}

  Celtic Games is an event that remembers days long past when kingdoms
  fought for control, power, riches, and territory; when kings and
  queens chose their champions to limit loss of life upon the
  battlefield. Will you be brave enough, strong enough, quick enough,
  or witty enough to take the coveted crown for your own and become
  King or Queen of \fpg Celtic Games.

  Lay on --- and may the most skilled sword claim victory!

  All ages are required to sign a consent \& liability form. Ages 5-12
  require a parent present; ages 5-17 require parental consent.

  Please bring a water gun with you on Friday for the Water Seige.

  \subsection{Coming Of Age}

  A Coming of Age ceremony is available for anyone who will be between
  the ages of 18 and 24 at the beginning of \fpg Beltane. If you or
  someone you know is interested in participating in (or helping with)
  the Coming of Age ordeal, contact Sage when he's over near Lodge 7,
  or at \texttt{Sage@star-hope.org} on or before March 15th.


  \subsection{Fire Circle} \label{fire}

  A parent (or adult guardian) must accompany any minors (17 and
  under) at the drum circle. Parents must mind their children's
  presence and safety. Children under 10 are allowed to stay after
  midnight, but may not dance around the fire then, for their safety
  and that of other dancers.

  Hooping or Poi spinning may be done in the designated area only.

  At fire circle, you may wear anything that you could wear on a
  public, clothing-non-optional beach.

  Neither smoking, nor glass containers, are permitted within the
  circle of torches.


  \subsubsection{Drum Circle Guidelines}
  Never touch another person's drum without permission.

  There is a very big fire. Falling-down drunkenness is inadvisable.

  Drummers and musicians should be in the same area. When you play,
  play with the rest of the drummers, not over them. Do not crowd out
  the dancers in the circle.

  Do not grope. whistle at, molest, or fondle the dancers. This goes
  for both male and female dancers.  It is not normally polite to
  ``plant'' yourself in front of a drummer and dance.

  Warming your drum around the fire is allowed, as long as you're
  walking around the circle clockwise -- with the flow of dancers --
  while holding your drum.

  There are fire tenders.  \emph{Leave the fire alone.}  This is
  Sacred Fire. Do not toss \emph{anything} into the fire. There are
  other fires at camp for you to play in.

  \subsubsection{Fire Temple Drum Altar}

  The Drum \& Dance Altar is a place for participants gain refreshment
  and sustenance (both physically and spiritually). It is built for
  the Community, and by the Community.

  Bring donation items to the drum circle before 10pm each night.
  Appreciated contributions include: blankets, rugs, tables, hay
  bales, candles, flowers, fruits, drinks, granola bars, and other
  midnight snacks.

  If you have any questions, stop by the Shiny Happy People Drum Tribe
  booth on Vendors' Row anytime during the festival.



  \subsection{Inipi Sweat Lodge} \label{sweatlodge}

  All should bring a blanket with which to cover the Lodge. The Lodge
  is made up of a frame constructed out of willow branches. The Lodge
  must be covered completely to hold in the heat and block out any
  light. Bringing your own blanket puts your personal energy into the
  construction of the Lodge.

  This ceremony is approached in a humble and peaceful manner. Around
  the sacred fire is the place to center ourselves and connect with
  the Creator prior to entering the Lodge.

  In the Apache tradition, we go into the Lodge as we came into the
  world, without clothing.  The Lodge represents the womb of the
  Mother Earth. We will wear a towel wrapped around us outside the
  Lodge and then drop the towels once inside. Pride, ego, and false
  modesty are left at the door. We enter as equals and are only
  concerned once inside with our own prayers. We are never concerned
  about what people look like or what others think about how we
  look. We enter the womb, die a symbolic spiritual death, and emerge
  reborn and renewed.

  As in all ceremonies, there needs to be a balance. An exchange of
  energy between the participants and the facilitator (water pourer)
  and fire tender must take place. Tobacco is a traditional gift, but
  whatever Spirit moves you to give is good as long as it comes from
  the heart.

  There is no charge for the ceremony --- just an exchange of energy
  for balance. Some people leave crafted items, objects from nature,
  books, CDs, etc. Some people leave nothing at all as they may be
  financially unable to give something. An exchange of energy can
  still take place through writing a poem, giving a big hug to the
  facilitator, etc. Whatever the case, it is ``all good'' if the gifts
  come from the heart.

  If you have any special items you want blessed with the energy of
  the ceremony, by all means, bring them and place them on the
  altar. People have brought pipes, ritual objects, and pictures of
  friends, family, and ancestors that have crossed over.



  \subsection{\fpg Wants You!}

  If you've got some time and want to give back to the community, visit
  the Ministry of Magick office (near the Blue Room) to sign up for a
  Community Service volunteer shift --- you could even win free
  admission to your next \fpg!

  Do you like making things happen? Giving back to your Community?
  Being a part of something Magikal? If you said ``Yes!'' to any of
  the above --- then we have a job for \emph{you}!

  \teg is always looking for good, reliable people to fill staff
  positions. There is a job for every type of person and ability, from
  sit-down jobs to physical activity, between \& during festivals.

  \subsubsection{The Perks} by becoming a Staff member your
  registration fee is reduced, and you get to come to festival 1-2
  days early! Additional perks include the Staff Ritual on Tuesday
  night, and the Annual Staff Retreat \& Meeting!

  If you are interested, talk to Medea and/or send an email to:
  \texttt{staffing@flapagan.org}


  \section{Life Events}

  \fpg makes it possible for our guests to participate in Life
  Event Rituals on site, including Handfastings, Weddings,
  Wiccanings, Initiations, Coming of Age Rituals, Elevations,
  Requiems, Cronings, Sagings, Coven Births, etc.

  If you are interested in having a Life Event at Festival, talk
  to Medea or email \texttt{lifeevents1@flapagan.org}

  \subsection{Calling all Covens \& Groups}

  Each Festival, the \FPG becomes a special community because you ---
  our guests --- bond together to create a family of like-minded
  people. To help promote this, \fpg encourages Covens, Groups, \&
  Friends to host the Meet \& Greets and the Main Rituals at each
  festival.

  If you are interested in hosting a future Meet \& Greet, or becoming
  the Marshall for a Parade, drop a note to
  \texttt{workshops@flapagan.org}.


  \subsubsection{Dragonwood Coven}
  holds open circles for sabbats and esbats on the Saturday
  closest. We have a beautiful outdoor circle in Pasco County.  For
  more information, call Ardy or Fred at 352-521-3647. During
  Festival, come see Dee the firetender.

  \subsubsection{Dogs for Heroes} is a non-profit organization that
  relies solely on donations and its volunteers to rescue and
  rehabilitate Pitty/Bully breeds, train them as Companion and Service
  dogs, and places them with disabled Veterans who suffer from
  service-related injuries/illness\footnote{such as Post Traumatic
    Stress Disorder (PTSD), Traumatic Brain Injuries (TBI), and
    Military Sexual Trauma (MST)}.  This gives them both a second
  chance at the best quality of life possible, together in a forever
  home.  Visit \texttt{dogsforheroes.org} or
  \texttt{facebook.com/DogsForHeroes}


  \subsection{Main Ritual Etiquette}

  While our opening and closing rituals tend to be more laid-back,
  \fpg's Main Ritual on Saturday is a magickal rite where we come
  together to worship the Gods, commune with the elements, and have a
  spiritual experience.

  Arrive at the ritual circle on time with your chairs or blankets.

  Prepare your mind as well as your body for ritual. Relax your mind and
  de-stress. Eat dinner, shower and change your clothes. Put on bug
  repellent. Mind the weather: you may need a sweater or blanket at
  Samhain.

  When entering the circle, follow the instructions of the ritual
  presenters, and enter the circle in a deocil (clockwise) motion
  unless told to do otherwise.

  Do not eat, drink, or smoke in the circle, or vicinity, before or
  during ritual. (Bottled water is allowed.)

  If you need to leave the circle for any reason, do so quietly and
  discreetly, and have someone cut you a door.

  Be courteous to the ritual presenters and other attendees.

  Do not speak nor comment during the ritual.

  \fpg's main rituals are hosted by Groups, Groves, Covens, and
  Churches within our Pagan community. These individuals put a lot of
  time and work into planning these rituals for all of our guests and
  staff to experience. We have many diverse paths and traditions, and
  sometimes these differ. Everyone who attends the ritual should be
  tolerant and respectful of the ritual presenters, and their work.


  \subsubsection{To present a Main Ritual} talk to any Board Member
  at Emerald City, or mail \texttt{tegbod@flapagan.org}.

  \section{Headliners}

  #include "handbook-headliner-bios.tex"

  \section{Musical Guests}

  #include "handbook-music-bios.tex"

  \section{Workshop Presenters}

  #include "handbook-workshop-bios.tex"

  \section{{\Large $\star$} Wednesday's Events}

  #include "Announcements-WED.tex"
  #include "Events-WED.tex"

  \section{{\Large $\star$} Thursday's Events}

  #include "Announcements-THU.tex"
  #include "Events-THU.tex"

  \section{{\Large $\star$} Friday's Events}

  #include "Announcements-FRI.tex"
  #include "Events-FRI.tex"

  \section{{\Large $\star$} Saturday's Events}

  #include "Announcements-SAT.tex"
  #include "Events-SAT.tex"

  \section{{\Large $\star$} Sunday's Event}
  #include "Events-SUN.tex"


\end{multicols}

\section{Vendors' Row}

\begin{multicols}{2}


  Visit our Vendors' Row, located near the Forest Gate and Sanctuary,
  to visit any of these fine vendors:

  #include "handbook-vendors.tex"


  \subsubsection{If you'd like to be an \fpg Vendor} it's easier than
  you might think! Send an eMail to \texttt{vendors@flapagan.org} for
  information.

  \section{Food}

  The local wildlife would like to eat your food just as much as you
  would! Take care with storing food, particularly if you're camping
  in a tent.

  \fpg strictly enforces Florida law: alcoholic beverages are for ages
  21+ \emph{only}.

  \subsection{The Bubbling Cauldron Meal Plan}

  The Bubbling Cauldron meal plan is a pre-paid dining service offered
  by \fpg and run by our staff.

  {\small Meal Plan availability is extremely limited, and must be
    pre-paid before Festival.}

  Meals are served in the Dining Hall. {\small Dinner is served early
    on Wed. so everyone may attend the Opening Ritual.}


  #include "insert-cauldron-meals.tex"

  \subsection{Food Vendors}

  You'll also find these food vendors near the Dining Hall:

  #include "handbook-food-vendors.tex"

  \section{Community Guidelines}

  \subsection{Arriving}

  The gates at \fpg open for guests at 3pm on Wednesday. Vendors may
  arrive at 9am on Wednesday.  Retreats by the Lake is not available
  for private camping in the days leading up to, or after, \fpg.

  All adults must show their photo ID when signing in, whether they
  are pre-registered or not, and sign a Release Agreement and Camera
  Usage Policy.

  Guests who arrive at night may be put up for the night in available
  space, so that they don't have to set up their camp in the dark.

  \subsection{Leaving}

  The gates at \fpg are open until midnight nightly. The gate will be
  open 1am-1:20, and 2am-2:20, to allow people to leave or
  return. Everyone is required to show their Festival wristbands.

  All guests must leave camp by 1pm on Sunday.

  \subsection{Contraband}

  Firearms or weapons are prohibited. (Ritual tools are allowed.)

  Illegal substances are prohibited.

  Neither pets nor familiars are allowed.

  \subsection{Need Help?}

  During the day, the Ministry of Magic Office is open, between the
  Dining Hall and Blue Room. After office hours, any staff member with
  a radio can page the Coordinator on Duty. The Board of Directors are
  found in the Emerald City. Guardians can be identified by staves
  with red flags, or lighted armbands at night.

  To speak to the \teg Board, to pass on compliments, concerns, or
  complaints, go to the Director's House --- Emerald City --- or email
  \texttt{tegbod@flapagan.org}.

  \subsection{Campsites}

  Campsites are first come, first served. No campsites are assigned. Areas
  near workshop and event sites are marked off for no camping. Do not camp
  outside of the fenced area, nor close to the shore of Tiger Lake or the
  lagoon; these are wildlife habitats.

  Sites are primitive. Electricity is not available --- except to
  those guests who have a legitimate medical need for it, such as for
  CPAP and Lymphapress machines. If you need this, see the Board of
  Directors in Emerald City for a special tag. You must provide your
  own extension cord. Access to electricity is for the medical
  equipment only, and not for luxuries such as lights, fans, heaters,
  cooking devices, nor other electronics. Extension cords may not run
  through windows nor doors.

  Tiki torches and fire bowls are allowed, but must always be attended
  while in use. Ground fires are \textsc{never} permitted in
  campsites. Discard ash in the metal can found near the entrance to
  the Forest, or near the fence, on the pool side of the lagoon. Do
  not discard ash on the ground, even if you've soaked it!

  Only take the space that your group needs, so other campers may have
  space too. Be polite and neighborly to your fellow festival-goers,
  so that all may have a good time.

  \subsection{Cabin \& Lodge Usage}

  Use of a Cabin or Lodge is for guests who paid for a cabin/lodge
  bunk only. All have central heat and air conditioning, and bathrooms
  with handicap showers. All are handicap accessible. Guests may
  request a specific cabin via \texttt{register@flapagan.org}. Each
  bunk requires a paid registration. Bunks are first come, first
  served.

  Specific bunks (such as lower bunks and single beds) can be reserved
  by guests with medical need or physical limits only. If you
  purchased a bunk, but plan to camp, allow other guests to take the
  lower bunks. Reserved bunks are marked with signs. Removing signs,
  or usurping a reserved bunk, is grounds for removal from the site.

  Do not smoke, vape, nor burn a candle nor incense inside any
  building. Do not cook in Cabins: this includes coffee pots, crock
  pots, hot plates, microwaves,  etc. Cooking has never been allowed
  in cabins.

  Seven lodges are also available, each with a bathroom and
  kitchen. Lodge guests may cook, but not with open flames or
  grilling. Bring your own cooking, eating, and cleaning supplies for
  the kitchen.

  Report problems with any buildings (such as broken lights, clogged
  toilets or other plumbing issues, or problems with the A/C) to the
  Office, or to the Coordinator on Duty after office hours.

  If any cabin/lodge is not left in a clean state, all attendees
  registered in that cabin may be restricted from renting cabin bunks
  in the future. Brooms and wipes are provided for your use.

  \subsection{Camera Use}

  Only guests and staff who have registered their cameras may take
  pictures --- and then, only of people who have given permission. This
  includes camera-phones and other recording devices.

  Obtain permission from a parent or guardian to photograph any child. It
  is never acceptable to photograph any nude child; this is illegal.

  Cameras, Camera Phones, or other recording devices are not allowed in
  the Fire Circle area (except the Staff Photographer).

  Do not photograph or record Main Ritual, BEFORE NOR DURING the
  ritual. If you wish to take pictures of the ritual presenters, altars,
  and guests or staff, you may do so after the ritual has ended and the
  circle is released, with their permission.

  Ask vendors before photographing anything in their vending tent.

  If you disobey these rules, your film or device (camera, phone) will be
  confiscated, to be returned (with the images of \fpg erased) at the end
  of Festival. You may also be removed from the site without a refund.


  \subsection{On-Site Facilities}

  Bathrooms, showers, and cold water spigots are available for guests'
  use. Be polite if you need to enter a campsite to access the water
  spigot. If your campsite is near a water spigot, allow access to the
  spigot to those who need it.

  Water must be carried back to your site for drinking and
  cleaning. Dishes and cooking materials must not be washed in the
  bathroom sinks, nor showers.

  Public showers can be found near the Pool and behind the
  Thunderdome, and in the back door to Cabin G. Public toilets can
  also be found in the Dining Hall and the Arts and Crafts Center.

  If you plan on cooking while at \fpg, bring your own cooking and
  cleaning materials. Grills are welcome. Cooking is not allowed in the
  cabins.

  Cell phones will work throughout most of the area, depending on the
  service provider. (AT\&T has the best coverage in the area.) WiFi
  Internet is not available. Charging stations are located in the
  Community Center.


  \subsection{Property}

  Mind your needs. Come to Festival with enough food, clothing, and
  supplies. Bring any medications or equipment you need.

  Mind your valuables. If you brought much cash, medicines, or
  valuable electronics, please bring cases with locks.


  \subsection{Wildlife}

  Hunting, fishing, plant foraging, and wood cutting are not allowed.

  We are fortunate to share our Festival with many creatures. The
  forest is home to many animals that we don't often see: bears, boars
  \& wild pigs, snakes, alligators, coyotes, and bobcats. If you see any
  animals, do not approach them. If an animal reacts to you --- you are
  too close to it. Leave the area immediately, and report their presence
  to a staff member or Guardian.

  Never feed a wild animal. Do not keep coolers or food inside, or
  next to, your tent. Dispose of your trash daily.

  \subsection{Alcohol \& Tobacco}

  Alcohol may only be possessed or consumed by adults of 21 years or
  older. Cigarettes and other tobacco or nicotine products (both
  smoking and smokeless) may only be used by adults of 18 years or
  older. This is the law. Do not give underage guests or staff alcohol
  or tobacco.

  Moderate your alcohol consumption to reasonable limits. If you are
  intoxicated and your behavior is unacceptable, you will be asked to
  return to your campsite.

  If you smoke (or vape), mind where you smoke. Smoke downwind from
  others. Do not smoke (nor vape) in any building, in workshop or
  event spaces, or near a ``No Smoking'' sign. Smoking is not
  permitted on the patio near the Dining Hall or outside the Blue Room
  and Community Center.

  \subsection{Sound \& Noise}

  \fpg has designated a quieter camping area near Emerald City. The
  site is large and open, and sound travels far. Drumming can be heard
  all over, sometimes as late at 3am. Mind your noise wherever you
  are. Lower the volume of your socialization after 1am. Be careful of
  noise when returning to your cabin. Do not speak loudly or shout
  near workshop or event venues when in use.


  \subsection{Respect}

  All guests and staff are responsible for themselves and any minors
  in their charge.  \fpg is a place where all Pagans can gather
  together to form a community and celebrate our faiths.

  Be tolerant of others regardless of their race, gender, sexual
  orientation, religious tradition, or any other way that a person
  differs from you.

  We require everyone to conduct themselves as mature, reasonable and
  responsible adults and be respectful of other people and their
  property.

  We do not take sides in Witch Wars or drama that may be plaguing a
  group outside: leave that drama at the gate, so that all may enjoy
  \fpg.

  Pagans are a loving group who don’t hesitate to embrace a stranger
  in a friendly hug or kiss.  This does not mean anyone has a right to
  invade your space, or behave objectionably around you, particularly
  regarding physical contact or sexual behavior.

  If someone encroaches on your personal space, it is your
  responsibility to communicate with that person about it. Be clear
  and forthcoming, but try to be kind. They may not mean to make you
  uncomfortable. You have the right to say ``no,'' and request they stop
  immediately.  Be firm if you must.

  ``No'' means ``No.'' ``No'' does not mean ``keep trying,'' nor ``ask later.''
  If you are told ``no,'' stay if the individual wishes to continue
  polite conversation; if they do not, then walk away. Respect their
  right to not speak further with you.

  Any harassment, stalking, or unwanted touch should be reported to
  the Guardians, Coordinator on Duty, or Board of Directors
  immediately, so we can assist you to resolve this issue (or report
  it to the authorities).


  \subsubsection{Adults.} This welcoming and accepting environment may
  allow you to forge new relationships. Consider the impact on
  existing relationships and feelings. Think before you act.

  Sex is permitted only between consenting adults. Inebriated people
  can not consent. No one under the age of 18 can consent to any
  sexual advances — neither physical, nor verbal.

  Limit sexual activity to private, personal space. Showers are not
  personal. Use discretion.  There may be children within earshot of
  you. Other adults may not wish to participate secondhand in your
  experience.

  Practice safer sex. Insist your partners do also.  Condoms are
  available at Guardian Camp.

  \subsection{Payment}

  Guests are expected to pay when they check-in. Guests with checks
  that bounce, or whose cards cannot be charged, are expected to pay
  any owed fees before leaving the site. Cash or money order will be
  the only acceptable forms of payment for these guests.

  Requests to cancel admission or meal plan cannot be made after the
  festival has begun.

\end{multicols}

{\Large Violation of these rules or instructions may be grounds for
  expulsion from the event without refund. }


\vspace{16pt}

{\Large The \TEG, Inc. Board of Directors and the \FPG Coordinators
  and Staff sincerely hope that you enjoy your time with us. }

\vspace{16pt}



\section{\teg Staff}

\begin{multicols}{2}

  \subsection{\teg Board of Directors}
  \begin{center}
    \begin{tabular}{rl}
      President & Ann Marie Augustino \\
      Vice President & Lady Rae Blackhood \\
      Secretary & Medea Athairn \\
      Treasurer & Teresa Hines \\
      Ombudsman & Paul Garrett \\
    \end{tabular}
  \end{center}

  \subsection{Division Coordinators}
  \begin{center}
    \begin{tabular}{rl}
      Guest Hearth      & Narissa \\
      Operations        & Mystral \\
      Marketing         & Akantha \\
      Registration      & Bobbi Jo \hspace{1in} \\
      Site              & Doug \\
      Staff Services    & Sandi \\
    \end{tabular}
    {\footnotesize D.C. in training, SMAC/Youth Services --- Akantha}
  \end{center}

  \subsection{Service Groups}
  \begin{center}
    \begin{tabular}{rl}
      Guardians & Sqrl \\
      Drummers  & Shiny Happy People\\
                & Drum Tribe \\
    \end{tabular}
  \end{center}

  \begin{center}
    \includegraphics[width=2in]{clansfolk.eps}
  \end{center}

  \subsection{\teg Elders}
  \begin{center}
    \begin{tabular}{l}
      Roger Paraselsu Coleman \\
      { \hspace{1in} \ldots \small Spiritual Advisor and Elder } \\
      Galan \\
      Thundar \\
      Guardian Bob \\
      Arachne \\
    \end{tabular}
  \end{center}

\end{multicols}


\subsection{Department Heads $($Lugals$)$}
\begin{center}
  \begin{tabular}{rl}
    Bubbling Cauldron    & Melody \& Shawn \\
    Comptroller          & Ray \& Jade \\
    Concert \& Sound     & Paul \\
    Design Team          & Medea, Diane \& Mystral \\
    Fire Circle          & Dee \\
    Gate                 & Tony \\
    Gungans              & Perseus \\
    Guest Hearth         & John ``Cowboy'' \\
    Kids' Realm          & Jennifer \\
    Parking              & Geoffrey \\
    Photography \& Web Design & Diane \\
    Registration         & Jim \& Amanda \\
    Ritual               & Roger Paraselsu Coleman \hspace{3em} \\
                           & \& Scion \\
    Site \& Strike       & Cliff \& Joe \\
    Staff Office         & Jessica \\
    Staffing             & Medea \\
    Store Runner         & Beth \\
    Taxi/Trolley         & Aurora \& Jim \\
    Technology           & Bruce--Robert \\
    Teen Forge           & \textit{To be announced} \\
    Trash                & Stew \\
    Tween Time           & Soren \\
    Vendors \& Workshops & Sage \\
  \end{tabular}
\end{center}

\vfill

{\tiny Typeset by \LaTeX --- \today --- Censorius Herald software \copyright 2013-\festyear{}}

% {\hminfamily\Large Notes?}      %

% \begin{tabular}{l}
%   \hspace{6.5in} \\
%   \hline \\
%   \hline \\
%   \hline \\
%   \hline \\
%   \hline \\
%   \hline \\
%   \hline \\
%   \hline \\
%   \hline \\
%   \hline \\
%   \hline \\
%   \hline \\
%   \hline \\
%   \hline \\
%   \hline \\
%   \hline \\
%   \hline \\
%   \hline \\
%   \hline \\
%   \hline \\
%   \hline \\
%   \hline \\
%   \hline \\
% \end{tabular}

\thispagestyle{headings}
\newpage
\pagestyle{empty}

\thispagestyle{empty}
\includepdf[pages={1},offset=-10mm 0mm,width=7in]{map.pdf}
\end{document}
